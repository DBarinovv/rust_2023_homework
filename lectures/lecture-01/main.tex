\documentclass[aspectratio=1610,t]{beamer}

\usepackage[english]{babel}
\usepackage{hyperref}
\usepackage{minted}
\usepackage{alltt}
\usepackage{amsmath}
\usepackage{graphicx}
\usepackage{xcolor}

\usetheme{metropolis}
\usemintedstyle{xcode}
\definecolor{codebg}{RGB}{247, 247, 246}
\setbeamercolor{background canvas}{bg=white}
\hypersetup{colorlinks,linkcolor=,urlcolor=orange}

\title{Lecture 1: Basics}
\date{February 21, 2023}
\author{Barinov Denis}
\institute{barinov.diu@gmail.com}

\begin{document}

% ----------------------------------------------------------------- %

\begin{frame}
\maketitle
\end{frame}

% ----------------------------------------------------------------- %

\begin{frame}[fragile]
\frametitle{Previous mistake}
\begin{itemize}
    \item Is really a pointer in compiled program.
    \item Cannot be \texttt{NULL}.
    \item Guaranties that the object is alive.
    \item There are \textt{\&} and \textt{\&mut} references.
\end{itemize}

\begin{minted}{rust}
    let mut x: i32 = 92;
    let r: &mut i32 = &mut x;  // Reference created explicitly
    *r += 1;                   // Explicit dereference
\end{minted}
\end{frame}

% ----------------------------------------------------------------- %

\begin{frame}[fragile]
\frametitle{Previous mistake}

\begin{minted}{rust}
  |
4 |     r += 1;                   // Explicit dereference
  |     -^^^^^
  |     |
  |     cannot use `+=` on type `&mut i32`
  |
help: `+=` can be used on `i32` if you dereference the left-hand side
  |
4 |     *r += 1;                   // Explicit dereference
  |     +
\end{minted}
\end{frame}

% ----------------------------------------------------------------- %

\begin{frame}[fragile]
\frametitle{Structures}
Structures are defined via \texttt{struct} keyword:

\begin{minted}[fontsize=\small]{rust}
    struct Example {
        oper_count: usize,
        data: Vec<i32>, // Note the trailing comma
    }
\end{minted}

Rust \textbf{do not} give any guarantees about memory representation by default. Even these structures can be different in memory!

\begin{minted}[fontsize=\small]{rust}
    struct A {
        x: Example,
    }

    struct B {
        y: Example,
    }
\end{minted}
\end{frame}

% ----------------------------------------------------------------- %

\begin{frame}[fragile]
\frametitle{Structures}
Let's add new methods to \texttt{Example}:

\begin{minted}[fontsize=\small]{rust}
    impl Example {
        // Associated
        pub fn new() -> Self {
            Self {
                oper_count: 0,
                data: Vec::new(),
            }
        }
    
        pub fn push(&mut self, x: i32) {
            self.oper_count += 1;
            self.data.push(x)
        }

        /* Next slide */
    }    
\end{minted}
\end{frame}

% ----------------------------------------------------------------- %

\begin{frame}[fragile]
\frametitle{Structures}
Let's add new methods to \texttt{Example}:

\begin{minted}[fontsize=\small]{rust}
    impl Example {
        /* Previous slide */

        pub fn oper_count(&self) -> usize {
            self.oper_count
        }
    
        pub fn eat_self(self) {
            println!("later on lecture :)")
        }
    }
\end{minted}

Note: you can have multiple \texttt{impl} blocks.
\end{frame}

% ----------------------------------------------------------------- %

\begin{frame}[fragile]
\frametitle{Structures}
Initialize a structure and use it:

\begin{minted}{rust}
    let mut x = Example {
        oper_count: 0,
        data: Vec::new(),
    };
    let y = Example::new();
    x.push(10);
    assert_eq!(x.oper_count(), 1);
\end{minted}
\end{frame}

% ----------------------------------------------------------------- %

\begin{frame}[fragile]
\frametitle{Simple example of generics}
What about being \textit{generic} over arguments?

\begin{minted}{rust}
    struct Example<T> {
        oper_count: usize,
        data: Vec<T>,
    }
\end{minted}
\end{frame}

% ----------------------------------------------------------------- %

\begin{frame}[fragile]
\frametitle{Simple example of generics}
What about being \textit{generic} over arguments?

\begin{minted}[fontsize=\small]{rust}
    impl<T> Example<T> {
        pub fn new() -> Self {
            Self {
                oper_count: 0,
                data: Vec::new(),
            }
        }

        pub fn push(&mut self, x: T) {
            self.oper_count += 1;
            self.data.push(x)
        }

        /* The rest is the same */
    } 
\end{minted}
\end{frame}

% ----------------------------------------------------------------- %

\begin{frame}[fragile]
\frametitle{Simple example of generics}
Initialize a structure and use it:

\begin{minted}[fontsize=\small]{rust}
    let mut x = Example<i32> {
        oper_count: 0,
        data: Vec::new(),
    };
    let y = Example::<i32>::new(); // ::<> called 'turbofish'
    let z: Example<i32> = Example {
        oper_count: 0,
        data: Vec::new(),
    };
    x.push(10);
    assert_eq!(x.oper_count(), 1);
\end{minted}
\end{frame}

% ----------------------------------------------------------------- %

\begin{frame}[fragile]
\frametitle{Turbofish}
Minimal C\texttt{++} code:
\begin{minted}{cpp}
template <int N>
class Terror {};

int main() {
    Clown<3> x;
}
\end{minted}
\end{frame}

% ----------------------------------------------------------------- %

\begin{frame}[fragile]
\frametitle{Turbofish}
\begin{minted}{cpp}
template <int N>
class Terror {};

int main() {
    Clown<3> x;
}
\end{minted}

\begin{minted}{bash}
<source>: In function 'int main()':
<source>:5:5: error: 'Clown' was not declared in this scope
    5 |     Clown<3> x;
      |     ^~~~~
<source>:5:14: error: 'x' was not declared in this scope
    5 |     Clown<3> x;
      |              ^
Compiler returned: 1
\end{minted}

\end{frame}

% ----------------------------------------------------------------- %

\begin{frame}[fragile]
\frametitle{\href{https://www.reddit.com/r/rust/comments/v4rir6/turbofish_why/}{Turbofish}}
\begin{minted}{cpp}
template <int N>
class Terror {};

int main() {
    // Clown<3> x;
    (Clown < 3) > x;
}
\end{minted}

\begin{minted}{bash}
<source>: In function 'int main()':
<source>:5:5: error: 'Clown' was not declared in this scope
    5 |     Clown<3> x;
      |     ^~~~~
<source>:5:14: error: 'x' was not declared in this scope
    5 |     Clown<3> x;
      |              ^
Compiler returned: 1
\end{minted}

\end{frame}

% ----------------------------------------------------------------- %

\begin{frame}[fragile]
\frametitle{Conditions and loops: \texttt{if}, \texttt{while}, \texttt{for}, \texttt{loop}}
\begin{minted}{rust}
    let mut x = 2;
    if x == 2 { // No braces in Rust
        x += 2;
    }
    while x > 0 { // No braces too
        x -= 1;
        println!("{x}");
    }
\end{minted}
\end{frame}

% ----------------------------------------------------------------- %

\begin{frame}[fragile]
\frametitle{Conditions and loops: \texttt{if}, \texttt{while}, \texttt{for}, \texttt{loop}}
\begin{minted}{rust}
    loop { // Just loop until 'return', 'break' or never return.
        println!("I'm infinite!");
        x += 1;
        if x == 10 {
            println!("I lied...");
            break
        }
    }
\end{minted}
\end{frame}

% ----------------------------------------------------------------- %

\begin{frame}[fragile]
\frametitle{Conditions and loops: \texttt{if}, \texttt{while}, \texttt{for}, \texttt{loop}}
This works in any other scope, for instance in \texttt{if}'s:

\begin{minted}{rust}
    let y = 42;
    let x = if y < 42 {
        345
    } else {
        y + 534
    }
\end{minted}
\end{frame}

% ----------------------------------------------------------------- %

\begin{frame}[fragile]
\frametitle{Conditions and loops: \texttt{if}, \texttt{while}, \texttt{for}, \texttt{loop}}
In Rust, we can break with a value from \texttt{while} and \texttt{loop}!

\begin{minted}{rust}
    let mut counter = 0;
    let result = loop {
        counter += 1;
        if counter == 10 {
            break counter * 2;
        }
    };
    assert_eq!(result, 20);
\end{minted}

Default \texttt{break} is just \texttt{break ()}.
\end{frame}

% ----------------------------------------------------------------- %

\begin{frame}[fragile]
\frametitle{\texttt{Inhabited type \texttt{!}}}
Rust always requires to return something correct.

\begin{minted}{rust}
    // error: mismatched types
    // expected `i32`, found `()`
    fn func() -> i32 {}
\end{minted}

How does this code work?

\begin{minted}{rust}
    fn func() -> i32 {
        unimplemented!("not ready yet")
    }
\end{minted}
\end{frame}

% ----------------------------------------------------------------- %

\begin{frame}[fragile]
\frametitle{\texttt{Inhabited type \texttt{!}}}
Rust always requires to return something correct.

\begin{minted}{rust}
    // error: mismatched types
    // expected `i32`, found `()`
    fn func() -> i32 {}
\end{minted}

How does this code work?

\begin{minted}{rust}
    fn func() -> i32 {
        unimplemented!("not ready yet")
    }
\end{minted}

\textbf{Return type that is never constructed}: \texttt{!}.
\end{frame}

% ----------------------------------------------------------------- %

\begin{frame}[fragile]
\frametitle{\texttt{Inhabited type \texttt{!}}}
\textbf{Return type that is never constructed}: \texttt{!}

Same as:

\begin{minted}{rust}
    enum Test {} // empty, could not be constructed
\end{minted}

\texttt{loop} without any \texttt{break} returns \texttt{!}
\end{frame}

% ----------------------------------------------------------------- %

\begin{frame}[fragile]
\frametitle{Conditions and loops: \texttt{if}, \texttt{while}, \texttt{for}, \texttt{loop}}
Or break on outer \texttt{while}, \texttt{for} or \texttt{loop}:

\begin{minted}[fontsize=\small]{rust}
    'outer: loop {
        println!("Entered the outer loop");
        'inner: for _ in 0..10 {
            println!("Entered the inner loop");

            // This would break only the inner loop
            // break;

            // This breaks the outer loop
            break 'outer;
        }
        println!("This point will never be reached");
    }
    println!("Exited the outer loop");
\end{minted}
\end{frame}

% ----------------------------------------------------------------- %

\begin{frame}[fragile]
\frametitle{Conditions and loops: \texttt{if}, \texttt{while}, \texttt{for}, \texttt{loop}}
Time for \texttt{for} loops!

\begin{minted}{rust}
    for i in 0..10 {
        println!("{i}");
    }
    for i in 0..=10 {
        println!("{i}");
    }
    for i in [1, 2, 3, 4] {
        println!("{i}");
    }
\end{minted}
\end{frame}

% ----------------------------------------------------------------- %

\begin{frame}[fragile]
\frametitle{Conditions and loops: \texttt{if}, \texttt{while}, \texttt{for}, \texttt{loop}}
Time for \texttt{for} loops!

\begin{minted}{rust}
    let vec = vec![1, 2, 3, 4];
    for i in &vec { // By reference
        println!("{i}");
    }
    for i in vec { // Consumes vec; will be discussed later
        println!("{i}");
    }
\end{minted}
\end{frame}

% ----------------------------------------------------------------- %

\begin{frame}[fragile]
\frametitle{Enumerations}
Enumerations are one of the best features in Rust :)

\begin{minted}{rust}
    enum MyEnum {
        First,
        Second,
        Third, // Once again: trailing comma
    }
    enum OneMoreEnum<T> {
        Ein(i32),
        Zwei(u64, Example<T>),
    }

    let x = MyEnum::First;
    let y: MyEnum = MyEnum::First;
    let z = OneMoreEnum::Zwei(42, Example::<usize>::new());
\end{minted}
\end{frame}

% ----------------------------------------------------------------- %

\begin{frame}[fragile]
\frametitle{Enumerations}
You can create custom functions for \texttt{enum}:

\begin{minted}{rust}
    enum MyEnum {
        First,
        Second,
        Third, // Once again: trailing comma
    }

    impl MyEnum {
        // ...
    }
\end{minted}
\end{frame}

% ----------------------------------------------------------------- %

\begin{frame}[fragile]
\frametitle{Enumerations: Option and Result}
In Rust, there's two important enums in \texttt{std}, used for error handling:

\begin{minted}{rust}
    enum Option<T> { 
        Some(T),
        None,
    }

    enum Result<T, E> {
        Ok(T),
        Err(E),
    }
\end{minted}

We will discuss them a bit later
\end{frame}

% ----------------------------------------------------------------- %

\begin{frame}[fragile]
\frametitle{Match}
\texttt{match} is one of things that will help you to work with \texttt{enum}.

\begin{minted}{rust}
    let x = MyEnum::First;
    match x {
        MyEnum::First => println!("First"),
        MyEnum::Second => {
            for i in 0..5 { println!("{i}"); }
            println!("Second");
        },
        _ => println!("Matched something!"),
    }
\end{minted}
\end{frame}

% ----------------------------------------------------------------- %

\begin{frame}[fragile]
\frametitle{The \texttt{\_} symbol}
\begin{itemize}
    \item \texttt{\_} matches everything in \texttt{match} (called wildcard).
    \item Used for inference sometimes:
    \begin{minted}{rust}
    // Rust does not know here to what type
    // you want to collect
    let mut vec: Vec<_> = (0..10).collect();
    vec.push(42u64);
    \end{minted}
    \item And to make a variable unused:
    \begin{minted}{rust}
    let _x = 10;
    // No usage of _x, no warnings!
    \end{minted}
\end{itemize}
\end{frame}

% ----------------------------------------------------------------- %

\begin{frame}[fragile]
\frametitle{Match}
\texttt{match} can match multiple objects at a time:

\begin{minted}{rust}
    let x = OneMoreEnum::<i32>::Ein(2);
    let y = MyEnum::First;
    match (x, y) {
        (OneMoreEnum::Ein(a), MyEnum::First) => {
            println!("Ein! - {a}");
        },
        // Destructuring
        (OneMoreEnum::Zwei(a, _), _) => println!("Zwei! - {a}"),
        _ => println!("oooof!"),
    }
\end{minted}
\end{frame}

% ----------------------------------------------------------------- %

\begin{frame}[fragile]
\frametitle{Match}
There's feature to match different values with same code:

\begin{minted}{rust}
    let number = 13;
    match number {
        1 => println!("One!"),
        2 | 3 | 5 | 7 | 11 => println!("This is a prime"),
        13..=19 => println!("A teen"),
        _ => println!("Ain't special"),
    }
\end{minted}
\end{frame}

% ----------------------------------------------------------------- %

\begin{frame}[fragile]
\frametitle{Match}
And we can apply some additional conditions called guards:

\begin{minted}{rust}
    let pair = (2, -2);
    println!("Tell me about {:?}", pair);
    match pair {
        (x, y) if x == y => println!("These are twins"),
        // The ^ `if condition` part is a guard
        (x, y) if x + y == 0 => println!("Antimatter, kaboom!"),
        (x, _) if x % 2 == 1 => println!("The first one is odd"),
        _ => println!("No correlation..."),
    }
\end{minted}
\end{frame}

% ----------------------------------------------------------------- %

\begin{frame}[fragile]
\frametitle{Match}
Match is an expression too:

\begin{minted}{rust}
    let x = 13;
    let res = match x {
        13 if foo() => 0,
        // You have to cover all of the possible cases
        13 => 1,
        _ => 2,
    };
\end{minted}
\end{frame}

% ----------------------------------------------------------------- %

\begin{frame}[fragile]
\frametitle{Match}
Ignoring the rest of the tuple:

\begin{minted}[fontsize=\small]{rust}
    let triple = (0, -2, 3);
    println!("Tell me about {:?}", triple);
    match triple {
        (0, y, z) => {
            println!("First is `0`, `y` is {y}, and `z` is {z}")
        },
        // `..` can be used to ignore the rest of the tuple
        (1, ..) => {
            println!("First is `1` and the rest doesn't matter")
        },
        _ => {
            println!("It doesn't matter what they are")
        },
    }
\end{minted}
\end{frame}

% ----------------------------------------------------------------- %

\begin{frame}[fragile]
\frametitle{Match}
Let's define a struct:

\begin{minted}[fontsize=\small]{rust}
    struct Foo {
        x: (u32, u32),
        y: u32,
    }

    let foo = Foo { x: (1, 2), y: 3 };
\end{minted}
\end{frame}

% ----------------------------------------------------------------- %

\begin{frame}[fragile]
\frametitle{Match}
Destructuring the struct:

\begin{minted}{rust}
    match foo {
        Foo { x: (1, b), y } => {
            println!("First of x is 1, b = {},  y = {} ", b, y);
        },
        Foo { y: 2, x: i } => {
            println!("y is 2, i = {:?}", i);
        },
        Foo { y, .. } => { // ignoring some variables:
            println!("y = {}, we don't care about x", y)
        },
        // Foo { y } => println!("y = {}", y),
        // error: pattern does not mention field `x`
    }
\end{minted}
\end{frame}

% ----------------------------------------------------------------- %

\begin{frame}[fragile]
\frametitle{Match}
Binding values to names:

\begin{minted}{rust}
    match age() {
        0 => println!("I haven't celebrated my birthday yet"),
        n @ 1..=12 => println!("I'm a child of age {n}"),
        n @ 13..=19 => println!("I'm a teen of age {n}"),
        n => println!("I'm an old person of age {n}"),
    }
\end{minted}
\end{frame}

% ----------------------------------------------------------------- %

\begin{frame}[fragile]
\frametitle{Match}
Binding values to names + arrays:

\begin{minted}{rust}
    let s = [1, 2, 3, 4];
    let mut t = &s[..]; // or s.as_slice()
    loop {
        match t {
            [head, tail @ ..] => {
                println!("{head}");
                t = &tail;
            }
            _ => break,
        }  
    } // outputs 1\n2\n\3\n4\n
\end{minted}
\end{frame}

% ----------------------------------------------------------------- %

\begin{frame}[fragile]
\frametitle{\texttt{if let}}
Sometimes we need only one enumeration variant to do something. Can we write it in a better way?

\begin{minted}{rust}
    let optional = Some(7);
    match optional {
        Some(i) => {
            println!("It's Some({i})");
        },
        _ => {},
        // ^ Required because `match` is exhaustive
    };
\end{minted}
\end{frame}

% ----------------------------------------------------------------- %

\begin{frame}[fragile]
\frametitle{\texttt{if let}}
Sometimes we need only one enumeration variant to do something. Can we write it in a better way?

\begin{minted}{rust}
    let optional = Some(7);
    if let Some(i) = optional {
        println!("It's Some({i})");
    }
\end{minted}
\end{frame}

% ----------------------------------------------------------------- %

\begin{frame}[fragile]
\frametitle{\texttt{while let}}
Same with \texttt{while}:

\begin{minted}{rust}
    let mut optional = Some(0);
    while let Some(i) = optional {
        if i > 9 {
            println!("Greater than 9, quit!");
            optional = None;
        } else {
            println!("`i` is `{i}`. Try again.");
            optional = Some(i + 1);
        }
    }
\end{minted}
\end{frame}

% ----------------------------------------------------------------- %

\begin{frame}[fragile]
\frametitle{Enumerations}
Let's dive into details

\begin{itemize}
    \item To identify the variant, we store some \textit{bits} in fields of enum. These bits are called \textit{discriminant}
    \item The count of bits is exactly as many as needed to keep the number of variants
    \item These bits are stored in unused bits of enumeration in another field. (compiler optimizations!)
\end{itemize}
\end{frame}

% ----------------------------------------------------------------- %

\begin{frame}[fragile]
\frametitle{Enumerations}

\begin{minted}{rust}
    enum Test {
        First(bool),
        Second,
        Third,
        Fourth,
    }
    assert_eq!(
        std::mem::size_of::<Test>(), 1
    );
    assert_eq!(
        std::mem::size_of::<Option<Box<i32>>>(), 8
    );
\end{minted}
\end{frame}

% ----------------------------------------------------------------- %

\begin{frame}[fragile]
\frametitle{Vector}
\begin{minted}{rust}
    let mut xs = vec![1, 2, 3];
    // To declare vector with same element and
    // specific count of elements, write
    // vec![42; 113];
    xs.push(4);
    assert_eq!(xs.len(), 4);
    assert_eq!(xs[2], 3);
\end{minted}
\end{frame}

% ----------------------------------------------------------------- %

\begin{frame}[fragile]
\frametitle{Slices}
We can create a slice to a vector or array. A slice is a contiguous sequence of elements in a collection.

\begin{minted}{rust}
    let a = [1, 2, 3, 4, 5];
    let slice1 = &a[1..4];
    let slice2 = &slice1[..2];
    assert_eq!(slice1, &[2, 3, 4]);
    assert_eq!(slice2, &[2, 3]);
\end{minted}
\end{frame}

% ----------------------------------------------------------------- %

\begin{frame}[fragile]
\frametitle{Panic!}
In Rust, when we encounter an unrecoverable error, we \texttt{panic!}

\begin{minted}{rust}
    let x = 42;
    if x == 42 {
        panic!("The answer!")
    }
\end{minted}

There are some useful macros that \texttt{panic!}

\begin{itemize}
    \item \texttt{unimplemented!}
    \item \texttt{unreachable!}
    \item \texttt{todo!}
    \item \texttt{assert!}
    \item \texttt{assert\_eq!}
\end{itemize}
\end{frame}

% ----------------------------------------------------------------- %

\begin{frame}[fragile]
\frametitle{\texttt{println!}}
The best tool for debugging, we all know. 

\begin{minted}[fontsize=\small]{rust}
    let x = 42;
    println!("{x}");
    println!("The value of x is {}, and it's cool!", x);
    println!("{:04}", x); // 0042
    println!("{value}", value=x + 1); // 43
    let vec = vec![1, 2, 3];
    println!("{vec:?}");   // [1, 2, 3]
    println!("{:?}", vec); // [1, 2, 3]
    let y = (100, 200);
    println!("{:#?}", y);
    // (
    //     100,
    //     200,
    // )
\end{minted}
\end{frame}

% ----------------------------------------------------------------- %

\begin{frame}
\center\Huge\textbf{Borrow Checker}

\includegraphics[height=6cm,keepaspectratio]{images/check-failed.jpeg}
\end{frame}

% ----------------------------------------------------------------- %

\begin{frame}[fragile]
\frametitle{What's the problem, Rust?}
\begin{minted}{rust}
    let mut v = vec![1, 2, 3];
    let x = &v[0];
    v.push(4);
    println!("{}", x);
\end{minted}
\end{frame}

% ----------------------------------------------------------------- %

\begin{frame}[fragile]
\frametitle{What's the problem, Rust?}
\begin{minted}{rust}
    let mut v = vec![1, 2, 3];
    let x = &v[0];
    v.push(4);
    println!("{}", x);
\end{minted}

\begin{minted}{bash}
error[E0502]: cannot borrow `v` as mutable because it is also
borrowed as immutable
 --> src/main.rs:8:5
  |
7 |     let x = &v[0];
  |              - immutable borrow occurs here
8 |     v.push(4);
  |     ^^^^^^^^^ mutable borrow occurs here
9 |     println!("{}", x);
  |                    - immutable borrow later used here
\end{minted}
\end{frame}

% ----------------------------------------------------------------- %

\begin{frame}[fragile]
\frametitle{What's the problem, Rust?}
\begin{minted}{rust}
    fn sum(v: Vec<i32>) -> i32 {
        let mut result = 0;
        for i in v {
            result += i;
        }
        result
    }

    fn main() {
        let mut v = vec![1, 2, 3];
        println!("first sum: {}", sum(v));
        v.push(4);
        println!("second sum: {}", sum(v))
    }
\end{minted}
\end{frame}

% ----------------------------------------------------------------- %

\begin{frame}[fragile]
\frametitle{What's the problem, Rust?}
\begin{minted}{rust}
error[E0382]: borrow of moved value: `v`
  --> src/main.rs:12:5
   |
10 |     let mut v = vec![1, 2, 3];
   |         ----- move occurs because `v` has type `Vec<i32>`,
   |  which does not implement the `Copy` trait
11 |     println!("first sum: {}", sum(v));
   |                                   - value moved here
12 |     v.push(4);
   |     ^^^^^^^^^ value borrowed here after move
\end{minted}
\end{frame}

% ----------------------------------------------------------------- %

\begin{frame}[fragile]
\frametitle{Ownership rules}
\begin{itemize}
    \item Each value in Rust has a variable that's called it's \textit{owner}.
    \item There can be only one owner at a time.
    \item When the owner goes out of scope, the value will be dropped.
\end{itemize}
\end{frame}

% ----------------------------------------------------------------- %

\begin{frame}[fragile]
\frametitle{Ownership rules}
\begin{minted}{rust}
    fn main() {
        let s = vec![1, 4, 8, 8];
        let u = s;
        println!("{:?}", u);
        println!("{:?}", s); // This won't compile!
    }
\end{minted}
\end{frame}

% ----------------------------------------------------------------- %

\begin{frame}[fragile]
\frametitle{Ownership rules}
\begin{minted}[fontsize=\small]{rust}
    fn om_nom_nom(s: Vec<i32>) {
        println!("I have consumed {s:?}");
    }

    fn main() {
        let s = vec![1, 4, 8, 8];
        om_nom_nom(s);
        println!("{s:?}");
    }
\end{minted}
\end{frame}

% ----------------------------------------------------------------- %

\begin{frame}[fragile]
\frametitle{Ownership rules}
\begin{minted}[fontsize=\small]{rust}
    fn om_nom_nom(s: Vec<i32>) {
        println!("I have consumed {s:?}");
    }

    fn main() {
        let s = vec![1, 4, 8, 8];
        om_nom_nom(s);
        println!("{s:?}");
    }
\end{minted}

\begin{itemize}
    \item Each "owner" has the responsibility to clean up after itself.
    \item When you move \texttt{s} into \textit{om\_nom\_nom}, it becomes the owner of \texttt{s}, and it will free \texttt{s} when it’s no longer needed in that scope. \textit{Technically the \texttt{s} parameter in \textit{om\_nom\_nom} become the owner}.
    \item That means you can no longer use it in \textit{main}!
    \item In C\texttt{++}, we will create a copy!
\end{itemize}
\end{frame}

% ----------------------------------------------------------------- %

\begin{frame}[fragile]
\frametitle{Ownership rules}
Given what we just saw, how can the following be the valid syntax?

\begin{minted}{rust}
    fn om_nom_nom(n: u32) {
        println!("{} is a very nice number", n);
    }

    fn main() {
        let n: u32 = 42;
        let m = n;
        om_nom_nom(n);
        om_nom_nom(m);
        println!("{}", m + n);
    }
\end{minted}
\end{frame}

% ----------------------------------------------------------------- %

\begin{frame}[fragile]
\frametitle{Ownership rules}
\begin{itemize}
    \item Say you have a group of lawyers that are reviewing and signing a contract over Google Docs (just pretend it's true :) )
    \item What are some ground rules we'd need to set to avoid chaos?
    \item If someone modifies the contract before everyone else reviews/signs it, that's fine.
    \item But if someone modifies the contract while others are reviewing it, people might miss changes and think they're signing a contract that says something else.
    \item We should allow a single person to modify, or everyone to read, but not both.
\end{itemize}
\end{frame}

% ----------------------------------------------------------------- %

\begin{frame}[fragile]
\frametitle{Borrowing intuition}
\begin{itemize}
    \item I should be able to have as many "const" pointers to a piece of data that I like.
    \item However, if I have a "non-const" pointer to a piece of data at the same time, this could invalidate what the other const pointers are viewing. (e.g., they can become dangling pointers...)
    \item If I have at most one "non-const" pointer at any given time, this should be OK.
\end{itemize}
\end{frame}

% ----------------------------------------------------------------- %

\begin{frame}[fragile]
\frametitle{Borrowing}
\begin{itemize}
    \item We can have multiple shared (immutable) references at once (with no mutable references) to a value.
    \item We can have only one mutable reference at once. (no shared references to it)
    \item This paradigm pops up a lot in systems programming, especially when you have "readers" and "writers". In fact, you've already studied it in the course of Theory and Practice of Concurrency.
\end{itemize}
\end{frame}

% ----------------------------------------------------------------- %

\begin{frame}[fragile]
\frametitle{Borrowing}
\begin{itemize}
    \item The lifetime of a value starts when it's created and ends the \textit{last time it's used}
    \item Rust doesn't let you have a reference to a value that lasts longer than the value's lifetime
    \item Rust computes lifetimes at compile time using static analysis. (this is often an over-approximation!)
    \item Rust calls the special "drop" function on a value once its lifetime ends. (this is essentially a destructor)
\end{itemize}
\end{frame}

% ----------------------------------------------------------------- %

\begin{frame}[fragile]
\frametitle{Borrowing}
\begin{minted}{rust}
fn main() {
    let mut x = 5;
    let y = &mut x;
    
    println!("y = {y}");
    x = 42; // ok
    println!("x = {x}");
}
\end{minted}
\end{frame}

% ----------------------------------------------------------------- %

\begin{frame}[fragile]
\frametitle{Borrowing}
\begin{minted}{rust}
fn main() {
    let mut x = 5;
    let y = &mut x;

    x = 42; // not ok
    println!("y = {y}");
    println!("x = {x}");
}
\end{minted}
\end{frame}

% ----------------------------------------------------------------- %

\begin{frame}[fragile]
\frametitle{Borrowing}
\begin{minted}{rust}
fn main() {
    let x1 = 42;
    let y1 = Box::new(84);
    { // starts a new scope
        let z = (x1, y1);
        // z goes out of scope, and is dropped;
        // it in turn drops the values from x1 and y1
    }
    // x1's value is Copy, so it was not moved into z
    let x2 = x1;
    
    // y1's value is not Copy, so it was moved into z
    // let y2 = y1;
}
\end{minted}
\end{frame}

% ----------------------------------------------------------------- %

\begin{frame}[fragile]
\frametitle{\texttt{Option}\footnote{\href{https://doc.rust-lang.org/std/option/enum.Option.html}{\texttt{Option} documentation}} and \texttt{Result}\footnote{\href{https://doc.rust-lang.org/std/result/enum.Result.html}{\texttt{Result} documentation}}}
Let's remember their definitions:

\begin{minted}{rust}
    enum Option<T> { 
        Some(T),
        None,
    }

    enum Result<T, E> {
        Ok(T),
        Err(E),
    }
\end{minted}
\end{frame}

% ----------------------------------------------------------------- %

\begin{frame}[fragile]
\frametitle{\texttt{Option} API}
Matching \texttt{Option}:

\begin{minted}{rust}
    let result = Some("string");
    match result {
        Some(s) => println!("String inside: {s}"),
        None => println!("Ooops, no value"),
    }
\end{minted}
\end{frame}

% ----------------------------------------------------------------- %

\begin{frame}[fragile]
\frametitle{\texttt{Option} API}
Useful functions \texttt{.unwrap()} and \texttt{.expect()}:

\begin{minted}{rust}
    fn unwrap(self) -> T;
    fn expect(self, msg: &str) -> T;
\end{minted}
\end{frame}

% ----------------------------------------------------------------- %

\begin{frame}[fragile]
\frametitle{\texttt{Option} API}
Useful functions \texttt{.unwrap()} and \texttt{.expect()}:

\begin{minted}[fontsize=\small]{rust}
    let opt = Some(22022022);
    assert!(opt.is_some());
    assert!(!opt.is_none());
    assert_eq!(opt.unwrap(), 22022022);
    let x = opt.unwrap(); // Copy!

    let newest_opt: Option<i32> = None;
    // newest_opt.expect("I'll panic!");

    let new_opt = Some(Vec::<i32>::new());
    assert_eq!(new_opt.unwrap(), Vec::<i32>::new());
    // error[E0382]: use of moved value: `new_opt`
    // let x = new_opt.unwrap(); // Clone!
\end{minted}
\end{frame}
 
% ----------------------------------------------------------------- %

\begin{frame}[fragile]
\frametitle{\texttt{Option} API}
We have a magic function: \begin{minted}[fontsize=\small]{rust}
    fn as_ref(&self) -> Option<&T>; // &self is &Option<T>
\end{minted}

Let's solve a problem:

\begin{minted}[fontsize=\small]{rust}
    let new_opt = Some(Vec::<i32>::new());
    assert_eq!(new_opt.unwrap(), Vec::<i32>::new());
    // error[E0382]: use of moved value: `new_opt`
    // let x = new_opt.unwrap(); // Clone!

    let opt_ref = Some(Vec::<i32>::new());
    assert_eq!(new_opt.as_ref().unwrap(), &Vec::<i32>::new());
    let x = new_opt.unwrap(); // We used reference!
    // There's also .as_mut() function
\end{minted}

That means if type implements \texttt{Copy}, \texttt{Option} also implements \texttt{Copy}.
\end{frame}
 
% ----------------------------------------------------------------- %

\begin{frame}[fragile]
\frametitle{\texttt{Option} API}
We can map \texttt{Option<T>} to \texttt{Option<U>}:

\begin{minted}[fontsize=\small]{rust}
    fn map<U, F>(self, f: F) -> Option<U>;
\end{minted}

Example:

\begin{minted}[fontsize=\small]{rust}
    let maybe_some_string = Some(String::from("Hello, World!"));
    // `Option::map` takes self *by value*,
    // consuming `maybe_some_string`
    let maybe_some_len = maybe_some_string.map(|s| s.len());
    assert_eq!(maybe_some_len, Some(13));
\end{minted}
\end{frame}

% ----------------------------------------------------------------- %

\begin{frame}[fragile]
\frametitle{\texttt{Option} API}
There's \textbf{A LOT} of different \texttt{Option} functions, enabling us to write beautiful functional code:

\begin{minted}{rust}
    fn map_or<U, F>(self, default: U, f: F) -> U;
    fn map_or_else<U, D, F>(self, default: D, f: F) -> U;
    fn unwrap_or(self, default: T) -> T;
    fn unwrap_or_else<F>(self, f: F) -> T;
    fn and<U>(self, optb: Option<U>) -> Option<U>;
    fn and_then<U, F>(self, f: F) -> Option<U>;
    fn or(self, optb: Option<T>) -> Option<T>;
    fn or_else<F>(self, f: F) -> Option<T>;
    fn xor(self, optb: Option<T>) -> Option<T>;
    fn zip<U>(self, other: Option<U>) -> Option<(T, U)>;
\end{minted}

It's recommended for you to study the documentation and try to avoid \texttt{match} where possible.
\end{frame}

% ----------------------------------------------------------------- %

\begin{frame}[fragile]
\frametitle{\texttt{Option} and ownership}
There's two cool methods to control ownership of the value inside:

\begin{minted}{rust}
    fn take(&mut self) -> Option<T>;
    fn replace(&mut self, value: T) -> Option<T>;
    fn insert(&mut self, value: T) -> &mut T;
\end{minted}

The first one takes the value out of the \texttt{Option}, leaving a \texttt{None} in its place.

The second one replaces the value inside with the given one, returning \texttt{Option} of the old value.

The third one inserts a value into the \texttt{Option}, then returns a mutable reference to it.
\end{frame}

% ----------------------------------------------------------------- %

\begin{frame}[fragile,c]
\frametitle{\texttt{Option} API and ownership: \texttt{take}}
\begin{minted}[fontsize=\small]{rust}
    struct Node<T> {
        elem: T,
        next: Option<Box<Node<T>>>,
    }

    pub struct List<T> {
        head: Option<Box<Node<T>>>,
    }

    impl<T> List<T> {
        pub fn pop(&mut self) -> Option<T> {
            self.head.take().map(|node| {
                self.head = node.next;
                node.elem
            })
        }
    }
\end{minted}
\end{frame}

% ----------------------------------------------------------------- %

\begin{frame}[fragile]
\frametitle{\texttt{Option} and optimizations}
Rust guarantees to optimize the following types \texttt{T} such that \texttt{Option<T>} has the same size as \texttt{T}:

\begin{itemize}
    \item \texttt{Box<T>}
    \item \texttt{\&T}
    \item \texttt{\&mut T}
    \item \texttt{fn}, \texttt{extern "C" fn}
    \item \texttt{\#[repr(transparent)]} struct around one of the types in this list.
    \item \texttt{num::NonZero*}
    \item \texttt{ptr::NonNull<T>}
\end{itemize}

This is called the ``null pointer optimization'' or NPO.
\end{frame}

% ----------------------------------------------------------------- %

\begin{frame}[fragile]
\frametitle{\texttt{Result}}
Functions return Result whenever errors are expected and recoverable. In the \texttt{std} crate, \texttt{Result} is most prominently used for I/O.

\textbf{Results must be used!} A common problem with using return values to indicate errors is that it is easy to ignore the return value, thus failing to handle the error. \texttt{Result} is annotated with the \texttt{\#[must\_use]} attribute, which will cause the compiler to issue a warning when a Result value is ignored.\footnote{\href{http://joeduffyblog.com/2016/02/07/the-error-model/}{The Error Model}}
\end{frame}

% ----------------------------------------------------------------- %

\begin{frame}[fragile]
\frametitle{\texttt{Result} API}
We can \texttt{match} it as a regular \texttt{enum}:

\begin{minted}{rust}
    let version = Ok("1.1.14");
    match version {
        Ok(v) => println!("working with version: {:?}", v),
        Err(e) => println!("error: version empty"),
    }
\end{minted}
\end{frame}

% ----------------------------------------------------------------- %

\begin{frame}[fragile]
\frametitle{\texttt{Result} API}
We have pretty the same functionality as in \texttt{Option}:

\begin{minted}[fontsize=\small]{rust}
    fn is_ok(&self) -> bool;
    fn is_err(&self) -> bool;
    fn unwrap(self) -> T;
    fn unwrap_err(self) -> E;
    fn expect_err(self, msg: &str) -> E;
    fn expect(self, msg: &str) -> T;
    fn as_ref(&self) -> Result<&T, &E>;
    fn as_mut(&mut self) -> Result<&mut T, &mut E>;
    fn map<U, F>(self, op: F) -> Result<U, E>;
    fn map_err<F, O>(self, op: O) -> Result<T, F>;
    // And so on
\end{minted}

It's recommended for you to study the documentation and try to avoid \texttt{match} where possible.
\end{frame}

% ----------------------------------------------------------------- %

\begin{frame}
\frametitle{Questions?}
\begin{center}
\includegraphics[width=\textwidth,height=7cm,keepaspectratio]{images/crab.jpg}
\end{center}
\end{frame}

% ----------------------------------------------------------------- %


\end{document}