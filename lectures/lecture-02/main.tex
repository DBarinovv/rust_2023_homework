\documentclass[aspectratio=1610,t]{beamer}

\usepackage[english]{babel}
\usepackage{hyperref}
\usepackage{minted}
\usepackage{alltt}
\usepackage{amsmath}
\usepackage{graphicx}
\usepackage{xcolor}

\usetheme{metropolis}
\usemintedstyle{xcode}
\definecolor{codebg}{RGB}{247, 247, 246}
\setbeamercolor{background canvas}{bg=white}
\hypersetup{colorlinks,linkcolor=,urlcolor=orange}

\title{Lecture 2: Traits}
\date{February 28, 2023}
\author{Barinov Denis}
\institute{barinov.diu@gmail.com}

\begin{document}

% ----------------------------------------------------------------- %

\begin{frame}
\maketitle
\end{frame}

% ----------------------------------------------------------------- %

\begin{frame}[fragile]
\frametitle{Operator \texttt{?}}
Consider the following structure:
\begin{minted}[fontsize=\small]{rust}
    struct Info {
        name: String,
        age: i32,
    }
\end{minted}
\end{frame}

% ----------------------------------------------------------------- %

\begin{frame}[fragile,c]
\frametitle{Operator \texttt{?}}
\begin{minted}[fontsize=\small]{rust}
fn write_info(info: &Info) -> io::Result<()> {
    let mut file = match File::create("my_best_friends.txt") {
        Err(e) => return Err(e),
        Ok(f) => f,
    };
    if let Err(e) = file
        .write_all(format!("name: {}\n", info.name)
        .as_bytes()) {
        return Err(e)
    }
    if let Err(e) = file
        .write_all(format!("age: {}\n", info.age)
        .as_bytes()) {
        return Err(e)
    }
    Ok(())
}
\end{minted}
\end{frame}

% ----------------------------------------------------------------- %

\begin{frame}[fragile]
\frametitle{Operator \texttt{?}}
We can use the \texttt{?} operator to make the code smaller!

\begin{minted}[fontsize=\small]{rust}
fn write_info(info: &Info) -> io::Result<()> {
    let mut file = File::create("my_best_friends.txt")?;
    file.write_all(format!("name: {}\n", info.name).as_bytes())?;
    file.write_all(format!("age: {}\n", info.age).as_bytes())?;
    Ok(())
}
\end{minted}

Beautiful, isn't it?

We can use it for \texttt{Option} too!
\end{frame}

% ----------------------------------------------------------------- %

\begin{frame}[fragile]
\frametitle{In this lecture}
\begin{itemize}
    \item Traits
    \item Exotically Sized Types
    \item Standard library traits
\end{itemize}
\end{frame}

% ----------------------------------------------------------------- %

\begin{frame}[c]
\centering\Huge\textbf{Traits}
\end{frame}

% ----------------------------------------------------------------- %

\begin{frame}[fragile]
\frametitle{Traits}

\center\includegraphics[height=6cm,keepaspectratio]{images/it's_time.jpg}

\end{frame}

% ----------------------------------------------------------------- %

\begin{frame}[fragile]
\frametitle{Traits}
In Rust, a trait is \textit{similar to} an interface in other languages. It's the way how we can define \textit{shared behavior} (i.e similarities between objects).

\begin{minted}[fontsize=\small]{rust}
    pub trait Animal {
        // No 'pub' keyword
        fn name(&self) -> String;
        fn noise(&self) -> String;
    }
\end{minted}

\end{frame}

% ----------------------------------------------------------------- %

\begin{frame}[fragile]
\frametitle{Traits}

\begin{minted}[fontsize=\small]{rust}
    pub trait Animal {
        fn name(&self) -> String;
        fn noise(&self) -> String;

        // Traits can provide default method definitions
        fn talk(&self) {
            println!("{} says {}", self.name(), self.noise());
        }
    }
\end{minted}

Let's define some structures and implement this trait for them.
\end{frame}

% ----------------------------------------------------------------- %

\begin{frame}[fragile]
\frametitle{Traits}
\begin{minted}{rust}
    pub struct Sheep {
        name: String,
    }

    impl Animal for Sheep {
        fn name(&self) -> String {
            self.name.clone()
        }

        fn noise(&self) -> String {
            "baaaaah!".to_string()
        }
    }
\end{minted}
\end{frame}

% ----------------------------------------------------------------- %

\begin{frame}[fragile]
\frametitle{Traits}
Usage example:

\begin{minted}{rust}
    let sheep = Sheep {
        name: "Dolly".to_string(),
    };
    assert_eq!(sheep.name(), "Dolly");
    sheep.talk();  // prints 'Dolly says baaaaah!'
\end{minted}
\end{frame}

% ----------------------------------------------------------------- %

\begin{frame}[fragile]
\frametitle{Traits}
\begin{minted}[fontsize=\small]{rust}
    pub struct Dog {
        name: String,
    }

    impl Animal for Dog {
        fn name(&self) -> String { self.name.clone() }

        fn noise(&self) -> String {
            "ruff!".to_string()
        }

        // Default trait methods can be overridden.
        fn talk(&self) {
            println!("Ruff! Don't call me doggo");
        }
    }
\end{minted}
\end{frame}


% ----------------------------------------------------------------- %

\begin{frame}[fragile]
\frametitle{Traits: \texttt{where} keyword}
And here we'll have some troubles:

\begin{minted}[fontsize=\small]{rust}
    pub trait Animal {
        fn name(&self) -> String;
        fn noise(&self) -> String;

        fn talk(&self) {
            // Note: this clones &Self, not Self!
            // let cloned = self.clone();

            let cloned = (*self).clone();
            println!("{} says {}", self.name(), self.noise());
        }
    }
\end{minted}
\end{frame}

% ----------------------------------------------------------------- %

\begin{frame}[fragile]
\frametitle{Traits: \texttt{where} keyword}
And here we'll have some troubles:

\begin{minted}[fontsize=\small]{rust}
    pub trait Animal {
        fn name(&self) -> String;
        fn noise(&self) -> String;

        fn talk(&self) {
            // error: no method named `clone` found for
            // type parameter `Self` in the current scope
            let cloned = (*self).clone();
            println!("{} says {}", self.name(), self.noise());
        }
    }
\end{minted}
\end{frame}

% ----------------------------------------------------------------- %

\begin{frame}[fragile]
\frametitle{Traits: \texttt{where} keyword}
To add bounds to the type, use \texttt{where} keyword.

\begin{minted}[fontsize=\small]{rust}
    pub trait Animal
    where
        Self: Clone
    {
        fn name(&self) -> String;
        fn noise(&self) -> String;

        fn talk(&self) {
            // Compiles just fine!
            // Note: this clones Self, not &Self!
            let cloned = self.clone();
            println!("{} says {}", cloned.name(), cloned.noise());
        }
    }
\end{minted}

By default, Rust doesn't expect anything* from types! You should provide bounds.
\end{frame}

% ----------------------------------------------------------------- %

\begin{frame}[fragile]
\frametitle{Traits: \texttt{where} keyword}
If we'll try to compile \texttt{Sheep} and \texttt{Dog} types, we'll see errors from the compiler:

\begin{minted}[fontsize=\small]{rust}
error[E0277]: the trait bound `Sheep: Clone` is not satisfied
  --> src/main.rs:22:6
   |
22 | impl Animal for Sheep {
   |      ^^^^^^ the trait `Clone` is not implemented for `Sheep`
   |
note: required by a bound in `Animal`
  --> src/main.rs:3:11
   |
1  | pub trait Animal
   |           ------ required by a bound-in this
2  | where
3  |     Self: Clone
   |           ^^^^^ required by this bound in `Animal`
\end{minted}
\end{frame}

% ----------------------------------------------------------------- %

\begin{frame}[fragile]
\frametitle{Traits: \texttt{where} keyword}
You can also write trait bounds in generics:

\begin{minted}[fontsize=\small]{rust}
    trait Strange1<T: Clone + Iterator> 
    where  // You're not able to do this without 'where'!
        T::Item: Clone,
    {
        fn smth(x: T);
    }

    trait Strange2<T> 
    where
        T: Clone + Iterator,
        T::Item: Clone,
    {
        fn smth(x: T);
    }
\end{minted}

Note that you can add trait bounds only to generics with \texttt{where}!
\end{frame}

% ----------------------------------------------------------------- %

\begin{frame}[fragile]
\frametitle{Supertraits}
Rust doesn't have ``inheritance'', but you can define a trait as being a superset of another trait:

\begin{minted}{rust}
    trait Shape { fn area(&self) -> f64; }
    trait Circle : Shape { fn radius(&self) -> f64; }
\end{minted}

Same as: 

\begin{minted}{rust}
    trait Shape { fn area(&self) -> f64; }
    trait Circle where Self: Shape { fn radius(&self) -> f64; }
\end{minted}



It's an example of declaring Shape to be a supertrait of Circle.
\end{frame}

% ----------------------------------------------------------------- %

\begin{frame}[fragile]
\frametitle{Supertraits}
We can use methods from another trait:

\begin{minted}{rust}
    trait Shape { fn area(&self) -> f64; }

    trait Circle: Shape {
        fn radius(&self) -> f64 {
            (self.area() /std::f64::consts::PI).sqrt()
        }
    }
\end{minted}
\end{frame}

% ----------------------------------------------------------------- %

\begin{frame}[fragile]
\frametitle{Supertraits}
Usage:

\begin{minted}{rust}
    fn print_area_and_radius<C: Circle>(c: C) {
        // Here we call the area method from
        // the supertrait `Shape` of `Circle`.
        println!("Area: {}", c.area());
        println!("Radius: {}", c.radius());
    }
\end{minted}
\end{frame}

% ----------------------------------------------------------------- %

\begin{frame}[fragile]
\frametitle{Fully Qualified Syntax}
What if types have multiple methods named the same way, and Rust cannot understand what method to call?

\begin{minted}[fontsize=\small]{rust}
    struct Form {
        username: String,
        age: u8,
    }

    trait UsernameWidget {
        fn get(&self) -> String;
    }

    trait AgeWidget {
        fn get(&self) -> u8;
    }
\end{minted}
\end{frame}

% ----------------------------------------------------------------- %

\begin{frame}[fragile]
\frametitle{Fully Qualified Syntax}
\begin{minted}{rust}
    impl UsernameWidget for Form {
        fn get(&self) -> String {
            self.username.clone()
        }
    }

    impl AgeWidget for Form {
        fn get(&self) -> u8 {
            self.age
        }
    }
\end{minted}
\end{frame}

% ----------------------------------------------------------------- %

\begin{frame}[fragile]
\frametitle{Fully Qualified Syntax}
Let's try to call \texttt{get}:

\begin{minted}{rust}
    let form = Form {
        username: "rustacean".to_owned(),
        age: 28,
    };

    println!("{}", form.get());
\end{minted}
\end{frame}

% ----------------------------------------------------------------- %

\begin{frame}[fragile]
\frametitle{Fully Qualified Syntax}
\begin{minted}[fontsize=\footnotesize]{rust}
error[E0034]: multiple applicable items in scope
  --> src/main.rs:35:25
   |
35 |     println!("{}", form.get());
   |                         ^^^ multiple `get` found
   |
note: candidate #1 is defined in an impl of the trait `UsernameWidget`
for the type `Form`
  --> src/main.rs:15:5
   |
15 |     fn get(&self) -> String {
   |     ^^^^^^^^^^^^^^^^^^^^^^^
note: candidate #2 is defined in an impl of the trait `AgeWidget`
for the type `Form`
  --> src/main.rs:21:5
   |
21 |     fn get(&self) -> u8 {
   |     ^^^^^^^^^^^^^^^^^^^
\end{minted}
\end{frame}

% ----------------------------------------------------------------- %

\begin{frame}[fragile]
\frametitle{Fully Qualified Syntax}
To solve the problem, one can call the method from a trait.

\begin{minted}{rust}
    let form = Form {
        username: "rustacean".to_owned(),
        age: 28,
    };

    // println!("{}", form.get());

    let username = UsernameWidget::get(&form);  // From trait
    assert_eq!("rustacean".to_owned(), username);
    let age = <Form as AgeWidget>::get(&form);  // FQS
    assert_eq!(28, age);
\end{minted}
\end{frame}

% ----------------------------------------------------------------- %

\begin{frame}[fragile]
\frametitle{Fully Qualified Syntax}

\begin{minted}{rust}
    let username = UsernameWidget::get(&form);
    let age = <Form as AgeWidget>::get(&form);
\end{minted}

\visible<2->{
    Actually, this one is called Fully Qualified Syntax (previously called universal function call syntax), and it's the most generic way of using methods.
}

\visible<3->{
    The angle bracket can be omitted if the type expression is a simple identifier (as in the first line), but is required for anything more complex. The syntax \texttt{<T as Trait>} means that we require that \texttt{T} implements the trait \texttt{Trait}, and the method after the double colon refers to a method from that trait implementation.
}
\end{frame}

% ----------------------------------------------------------------- %

\begin{frame}[fragile]
\frametitle{\texttt{impl} keyword}
What if you need to accept any type that implements some trait? You can do the following:

\begin{minted}{rust}
    fn func<T: MyTrait + Clone>(input: T) {
        // ...
    }
\end{minted}
\end{frame}

% ----------------------------------------------------------------- %

\begin{frame}[fragile]
\frametitle{\texttt{impl} keyword}
...Or use special syntax!

One argument:
\begin{minted}{rust}
    fn func(input: impl MyTrait + Clone) {
        // ...
    }
\end{minted}

Two arguments:
\begin{minted}{rust}
    fn func(input: &impl MyTrait, output: &impl MyTrait) {
        // ...
    }
\end{minted}

One complex argument:
\begin{minted}{rust}
    fn func(input: &(impl MyTrait + Clone)) {
        // ...
    }
\end{minted}

\end{frame}

% ----------------------------------------------------------------- %

\begin{frame}[fragile]
\frametitle{\texttt{impl} keyword}
\begin{minted}{rust}
fn func<T: Display + Clone, U: Clone + Debug>(t: &T, u: &U) -> i32 { ... }
\end{minted}
Same as:
\begin{minted}{rust}
fn func<T, U>(t: &T, u: &U) -> i32
where
    T: Display + Clone,
    U: Clone + Debug,
{ ... }
\end{minted}

\end{frame}

% ----------------------------------------------------------------- %

\begin{frame}[fragile]
\frametitle{Multiple \texttt{impl}}
What if we want to implement additional methods for a type depending on if it has an implementation of some trait?

\begin{minted}[fontsize=\small]{rust}
    pub enum Option<T> {
        // ...
    }

    impl<T> Option<T> {
        // ..
    }

    impl<T> Option<T>
    where
        T: Default
    {
        // ...
    }
\end{minted}
\end{frame}

% ----------------------------------------------------------------- %

\begin{frame}[fragile]
\frametitle{\texttt{where} and selection}
We can implement methods depending on whether the type has implementations of some traits.

\begin{minted}[fontsize=\small]{rust}
    pub enum Option<T> {
        // ...
    }

    impl<T> Option<T> {
        pub fn unwrap_or_default<T>(self) -> T
        where
            T: Default
        {
            // ...
        }
    }
\end{minted}
\end{frame}

% ----------------------------------------------------------------- %

\begin{frame}[c]
\centering\Huge\textbf{Exotically Sized Types}
\end{frame}

% ----------------------------------------------------------------- %

\begin{frame}[fragile]
\frametitle{Exotically Sized Types}
Most of the time, we expect types to have a statically known and positive size. This isn't always the case in Rust!

Currently, types can be:

\begin{itemize}
    \item ``Regular'' (no formal name as far as lecturer knows)
    \item Dynamically Sized Types, DST
    \item Zero Sized Types, ZST
    \item Empty Types
\end{itemize}
\end{frame}

% ----------------------------------------------------------------- %

\begin{frame}[fragile]
\frametitle{Dynamically Sized Types}
Dynamically Sized Type is a type which size is unknown at compile time.

There are only two kinds of DST's:

\begin{itemize}
    \item Slices, either regular such as \texttt{[u8]} and \texttt{str}.
    \item Trait objects, such as \texttt{dyn Trait}.
\end{itemize}
\end{frame}

% ----------------------------------------------------------------- %

\begin{frame}[fragile]
\frametitle{Dynamically Sized Types}
Dynamically Sized Type is a type which size is unknown at compile time.

There are only two kinds of DST's:

\begin{itemize}
    \item Slices, either regular such as \texttt{[u8]} and \texttt{str}.
    \item Trait objects, such as \texttt{dyn Trait}.
\end{itemize}

Such types \textbf{do not} implement \texttt{Sized} marker trait. \textbf{By default, Rust ``implements'' it for all types it can!}

\begin{minted}{rust}
    pub trait Sized {}
\end{minted}
\end{frame}

% ----------------------------------------------------------------- %

\begin{frame}[fragile]
\frametitle{Dynamically Sized Types: Slices}
Remember: Rust has strict type system. For instance, types \texttt{T} and \texttt{\&T} are \textbf{different}.

All that time we've written \texttt{\&str} instead of just \texttt{str} and that's for reason! Since the size of slice is not known at compile time, \texttt{str} is \textbf{unsized}, and it's a separate type.

Basically, \texttt{\&str} is just a pointer to the beginning of the slice and its length, and it means the reference to the slice is sized.

The same stands true for \texttt{[u8]}, \texttt{[i64]} and others.
\end{frame}

% ----------------------------------------------------------------- %

\begin{frame}[fragile]
\frametitle{Dynamically Sized Types: Trait objects}
Consider the following code:

\begin{minted}{rust}
    trait Hello {
        fn hello(&self);
    }

    fn func(arr: &[Hello]) {
        for i in arr {
            i.hello();
        }
    }
\end{minted}

Will it compile?

\visible<2->{
    \textbf{No}, since the compiler doesn't know which size every object that implements \texttt{Hello} have and therefore cannot put them in the slice.
}
\end{frame}

% ----------------------------------------------------------------- %

\begin{frame}[fragile]
\frametitle{Dynamically Sized Types: Trait objects}
Consider the following code:

\begin{minted}{rust}
    fn func<T: Hello>(arr: &[T]) {
        for i in arr {
            i.hello();
        }
    }
\end{minted}

Will it compile?

\visible<2->{
    \textbf{Yes}, since compiler knows which size every object have. It will generate unique instance of function for every \texttt{T}.
}

\visible<3->{
    But what if we need an array of objects that implement \texttt{Hello}?
}
\end{frame}

% ----------------------------------------------------------------- %

\begin{frame}[fragile]
\frametitle{Dynamically Sized Types: Trait objects}
\begin{minted}{rust}
    fn func(arr: &[&dyn Hello]) {
        for i in arr {
            i.hello();
        }
    }
\end{minted}

\begin{itemize}[leftmargin=0pt]
    \item<2-> Keyword \texttt{dyn} creates a \textbf{trait object}: some object that implements \texttt{Hello}.
    \item<3-> \texttt{dyn Hello} is also an \textbf{unsized} type, since we don't know the size of the object that implements it.
    \item<4-> \texttt{\&dyn Hello} consists of pointer to the structure and the pointer to the \textit{virtual table}, and it's sized. This reference is called \textbf{fat pointer}.
\end{itemize}
\end{frame}

% ----------------------------------------------------------------- %

\begin{frame}[fragile]
\frametitle{Dynamically Sized Types: Trait objects}
Trait objects can be stored at any pointers:

\begin{minted}{rust}
    impl Hello for &str {
        fn hello(&self) {
            println!("hello &str!");
        }
    }

    let x = "hello world";
    let r1: &dyn Hello = &x;
    let r2: Box<dyn Hello> = Box::new(x.clone());
\end{minted}
\end{frame}

% ----------------------------------------------------------------- %

\begin{frame}[fragile]
\frametitle{Dynamically Sized Types: Trait objects}
You cannot require more than one \textbf{non-auto} trait in trait objects, use supertraits instead.

\begin{minted}{rust}
    let x = "hello world";

    let r: &dyn Hello + World = &x; // World is some regular user trait,
                                    // it won't compile!
    
    trait HelloWorld: Hello + World {}
    impl HelloWorld for &str {
        // ...
    }

    // Will compile just fine
    let r: &dyn HelloWorld = &x;
\end{minted}
\end{frame}

% ----------------------------------------------------------------- %

\begin{frame}[fragile]
\frametitle{Dynamically Sized Types: Trait objects}
But you can require additional auto traits:

\begin{minted}{rust}
    trait X {
        // ...
    }

    fn test(x: Box<dyn X + Send>) {
        // ...
    }
\end{minted}
Auto traits: Send, Sync, Unpin, UnwindSafe, and RefUnwindSafe.
\end{frame}

% ----------------------------------------------------------------- %

\begin{frame}[fragile]
\frametitle{Trait objects: object safety}
Ok, let's compile the following code:

\begin{minted}{rust}
    fn test(x: Box<dyn Clone + Send>) {
        // ...
    }
\end{minted}
\end{frame}

% ----------------------------------------------------------------- %

\begin{frame}[fragile]
\frametitle{Trait objects: object safety}
\begin{minted}[fontsize=\small]{rust}
error[E0038]: the trait `Clone` cannot be made into an object
 --> src/main.rs:1:16
  |
1 | fn test(x: Box<dyn Clone + Send>) {
  |                ^^^^^^^^^^^^^^^^ `Clone` cannot be made
  |                                 into an object
  |
  = note: the trait cannot be made into an object because it
  requires `Self: Sized`
  = note: for a trait to be "object safe" it needs to allow
  building a vtable to allow the call to be resolvable dynamically
\end{minted}
\end{frame}

% ----------------------------------------------------------------- %

\begin{frame}[fragile]
\frametitle{Trait objects: object safety}
A trait is object safe if it has the following \href{https://doc.rust-lang.org/reference/items/traits.html#object-safety}{qualities}:

\begin{itemize}
    \item All supertraits must also be object safe
    \item Sized must not be a supertrait. In other words, it must not require Self: Sized
    \item It must not have any associated constants
    \item It must not have any associated types with generics
    \item About associated functions...
\end{itemize}
\end{frame}

% ----------------------------------------------------------------- %

\begin{frame}[fragile]
\frametitle{\texttt{impl dyn}}
We can implement methods for trait objects!

\begin{minted}[fontsize=\small]{rust}
    impl dyn Example {
        fn is_dyn(&self) -> bool {
            true
        }
    }

    struct Test {}
    impl Example for Test {}

    let x = Test {};
    let y: Box<dyn Example> = Box::new(Test {});
    
    x.is_dyn() // Won't compile
    y.is_dyn();
\end{minted}
\end{frame}

% ----------------------------------------------------------------- %

\begin{frame}[fragile]
\frametitle{Trait objects vs Generics}
\textbf{Question}: When to prefer Trait objects over generics and vice versa?

\begin{itemize}
    \item<2-> Trait objects produce less code and therefore prevent code bloating.
    \item<3-> But require to read the vtable.
    \item<4-> Generics are generally faster since they allow type-specific optimizations.
    \item<5-> But when there are many types using generic function, code becomes bigger and CPU cannot fit it all into memory.
    \item<6-> In this case, trait objects will be faster since single implementation would fit into cache line.
    \item<7-> \textbf{Answer}: only profiling can really help you with this question. 
\end{itemize}
\end{frame}

% ----------------------------------------------------------------- %

\begin{frame}[c]
\centering\Huge\textbf{Standard library traits}
\end{frame}

% ----------------------------------------------------------------- %

\begin{frame}[fragile]
\frametitle{Just a bit information about macros}
In the first lecture, we mentioned that macros are a way of code generation in Rust. We can also use or even write a macro that will generate an implementation of trait automatically - \texttt{derive}.

Such type of macros is called \textbf{procedural macros}, whereas macros such as \texttt{println!} are \textbf{declarative}.

We'll discuss this in more detail a little bit later.
\end{frame}

% ----------------------------------------------------------------- %

\begin{frame}[fragile]
\frametitle{\texttt{Default}}
Creates some default instance of \texttt{T}. Has a \texttt{\#[derive(Default)]} macro.

\begin{minted}{rust}
    pub trait Default {
        fn default() -> Self;
    }
\end{minted}
\end{frame}

% ----------------------------------------------------------------- %

\begin{frame}[fragile]
\frametitle{\texttt{Default}}
Many types in Rust have a constructor. However, this is specific to the type; Rust cannot abstract over  ``everything that has a \texttt{.new()} method''.

To allow this, the \texttt{Default} trait was conceived, which can be used with containers and other generic types (e.g. \texttt{Option::unwrap\_or\_default()}).

\textbf{Question}: why this trait is not derived by default?
\end{frame}

% ----------------------------------------------------------------- %

\begin{frame}[fragile]
\frametitle{\texttt{Clone}}
A trait for the ability to explicitly duplicate an object. Has a \texttt{\#[derive(Clone)]} macro.

\begin{minted}{rust}
    pub trait Clone {
        fn clone(&self) -> Self;

        // Note the default implementation!
        fn clone_from(&mut self, source: &Self) {
            *self = source.clone()
        }
    }
\end{minted}
\end{frame}

% ----------------------------------------------------------------- %

\begin{frame}[fragile]
\frametitle{\texttt{Copy}}
Types whose values can be duplicated simply by copying bits. Has a \texttt{\#[derive(Copy)]} macro.

It's \textbf{marker trait} and exists only to show the compiler that the type is special and can be copied by just copying bits of type representation.

\begin{minted}{rust}
    pub trait Copy: Clone {}
\end{minted}

By default, variable bindings have \textbf{``move semantics''}. However, if a type implements \texttt{Copy}, it instead has \textbf{``copy semantics''}.
\end{frame}

% ----------------------------------------------------------------- %

\begin{frame}
\frametitle{Summary}
\begin{itemize}
    \item Traits syntax
    \item Dyn, impl keywords
    \item Some standard traits
\end{itemize}
\end{frame}

% ----------------------------------------------------------------- %

\begin{frame}
\frametitle{Questions?}
\begin{center}
\includegraphics[width=\textwidth,height=7cm,keepaspectratio]{images/crab.jpg}
\end{center}
\end{frame}

% ----------------------------------------------------------------- %

\end{document}
