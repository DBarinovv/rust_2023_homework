\documentclass[aspectratio=1610,t]{beamer}

\usepackage[english]{babel}
\usepackage{hyperref}
\usepackage{minted}
\usepackage{alltt}
\usepackage{amsmath}
\usepackage{graphicx}
\usepackage{xcolor}
\usepackage[utf8]{inputenc}
\usepackage[T1]{fontenc}

\usetheme{metropolis}
\usemintedstyle{xcode}
\definecolor{codebg}{RGB}{247, 247, 246}
\setbeamercolor{background canvas}{bg=white}
\hypersetup{colorlinks,linkcolor=,urlcolor=orange}

\title{Lecture 4: Rewind}
\date{March 7, 2023}
\author{Barinov Denis}
\institute{barinov.diu@gmail.com}

\begin{document}

% ----------------------------------------------------------------- %

\begin{frame}
\maketitle
\end{frame}

% ----------------------------------------------------------------- %
\begin{frame}[c]
\centering\Huge\textbf{\texttt{String} and \texttt{\&str}}
\end{frame}

% ----------------------------------------------------------------- %

\begin{frame}[fragile]
\frametitle{\texttt{String}\footnote{\href{https://doc.rust-lang.org/std/string/struct.String.html}{\texttt{String} documentation}}}
A Rust way to store a string.

\begin{itemize}
    \item<2-> It's \textbf{UTF-8–encoded}.
    \item<3-> Growable like a \texttt{Vec}. It also made up of three components: a pointer to some bytes, a length, and a capacity. This even gives us many functions same to \texttt{Vec}.
    \item<4-> UTF-8 is a variable-width character encoding, so you cannot index it since it's UTF-8. To find N-th symbol, you should iterate over string, parsing code points.
\end{itemize}
\end{frame}

% ----------------------------------------------------------------- %

\begin{frame}[fragile,c]
\frametitle{\texttt{String} API}
\begin{minted}[fontsize=\small]{rust}
    struct String {
        vec: Vec<u8>,
    }

    impl String {
        fn new() -> String;
        fn with_capacity(capacity: usize) -> String;
        fn from_utf8(vec: Vec<u8>) -> Result<String, FromUtf8Error>;
        fn from_utf16(v: &[u16]) -> Result<String, FromUtf16Error>;
        fn into_bytes(self) -> Vec<u8>;
        fn as_bytes(&self) -> &[u8];
    }
\end{minted}
\end{frame}

% ----------------------------------------------------------------- %

\begin{frame}[fragile]
\frametitle{\texttt{String} in depth}
What will this code print?

\begin{minted}{rust}
    let s = String::from("привет"); // hello
    println!("{}", s.len());
\end{minted}
\end{frame}

% ----------------------------------------------------------------- %

\begin{frame}[fragile]
\frametitle{\texttt{String} in depth}
What will this code print?

\begin{minted}{rust}
    let s = String::from("привет"); // hello
    println!("{}", s.len());
\end{minted}

This outputs 12, since \texttt{.len()} gives count of \textit{bytes} in string.
\end{frame}

% ----------------------------------------------------------------- %


\begin{frame}[fragile]
\frametitle{\texttt{char} type}
\begin{minted}{rust}
    let mut chars = "é".chars();
    // U+00e9: 'latin small letter e with acute'
    assert_eq!(Some('\u{00e9}'), chars.next());
    assert_eq!(None, chars.next());

    let mut chars = "é".chars();
    // U+0065: 'latin small letter e'
    assert_eq!(Some('\u{0065}'), chars.next());
    // U+0301: 'combining acute accent'
    assert_eq!(Some('\u{0301}'), chars.next());
    assert_eq!(None, chars.next());
\end{minted}
\end{frame}

% ----------------------------------------------------------------- %

\begin{frame}[fragile]
\frametitle{\texttt{char} type}
The size of \texttt{char} is always 4 bytes:

\begin{minted}{rust}
    assert_eq!(std::mem::size_of::<char>(), 4);
\end{minted}
\end{frame}

% ----------------------------------------------------------------- %

\begin{frame}[fragile]
\frametitle{\texttt{\&str}}
\texttt{\&str} is a slice type of \texttt{String}, similar to \texttt{std::string\_view}. Just like:

\begin{minted}{rust}
    let vec = vec![1, 2, 3, 4];
    let vec_slice = &vec[1..3]; // &[2, 3]
    let s = String::from("hello");
    let s_slice = &s[1..3]; // "el"
\end{minted}
\end{frame}

% ----------------------------------------------------------------- %

\begin{frame}[fragile]
\frametitle{\texttt{\&str}}
Ok, let's take a UTF-8 slice!

\begin{minted}{rust}
    let s = String::from("привет"); // hello
    let s_slice = &s[1..3];
\end{minted}
\end{frame}

% ----------------------------------------------------------------- %

\begin{frame}[fragile]
\frametitle{\texttt{\&str}}
Ok, let's take a UTF-8 slice!

\begin{minted}{rust}
    let s = String::from("привет"); // hello
    let s_slice = &s[1..3];
    // thread 'main' panicked at 'byte index 1 is
    // not a char boundary; it is inside 'п' // h
    // (bytes 0..2) of `привет`' // hello
\end{minted}
\end{frame}

% ----------------------------------------------------------------- %

\begin{frame}[fragile]
\frametitle{\texttt{\&str}}
Ok, let's take a UTF-8 slice!

\begin{minted}{rust}
    let s = String::from("привет"); // hello
    let s_slice = &s[1..3];
    // thread 'main' panicked at 'byte index 1 is
    // not a char boundary; it is inside 'п' // h
    // (bytes 0..2) of `привет`' // hello
\end{minted}

That means \texttt{\&str} also have a UTF-8 invariant checked at runtime.
\end{frame}

% ----------------------------------------------------------------- %

\begin{frame}[fragile]
\frametitle{\texttt{\&str}}
As a string slice, \texttt{\&str} have most functions \texttt{String} have:

\begin{minted}{rust}
    fn as_bytes(&self) -> &[u8];
    fn chars(&self) -> Chars<'_>;
    fn trim(&self) -> &str;
    // And so on
\end{minted}
\end{frame}

% ----------------------------------------------------------------- %

\begin{frame}[fragile]
\frametitle{\texttt{\&str}}
All string constants are \texttt{\&str}.

\begin{minted}{rust}
    let s: &str = "Hello world!";
    let t1 = s.to_string();
    let t2 = s.to_owned(); // The same as t1
    let t3 = String::from(s); // The same as t1
\end{minted}
\end{frame}

% ----------------------------------------------------------------- %

\begin{frame}[c]
\centering\Huge\textbf{\texttt{Box} and \texttt{Rc}}
\end{frame}

% ----------------------------------------------------------------- %

\begin{frame}[fragile]
\frametitle{\texttt{Box}}
We are already familiar with \texttt{Box} type. Let's check one advanced function:

\begin{minted}{rust}
    fn leak<'a>(b: Box<T, A>) -> &'a mut T;
    fn into_raw(b: Box<T, A>) -> *mut T;
\end{minted}

Example:

\begin{minted}{rust}
    let x = Box::new(41);
    let static_ref: &'static mut usize = Box::leak(x);
    *static_ref += 1;
    assert_eq!(*static_ref, 42);
\end{minted}
\end{frame}

% ----------------------------------------------------------------- %

\begin{frame}[fragile]
\frametitle{\texttt{Box}}
\textbf{But stop!} Rust is the safe language, no memory unsafety, no undefined behavior, what's wrong!?

\visible<2->{
    \textit{In reality, when you're creating global objects or interacting with other languages, you \textbf{have to} leak objects. Moreover, it's \textbf{safe} to leak memory, just not good!}
}
\end{frame}

% ----------------------------------------------------------------- %

\begin{frame}[fragile]
\frametitle{\texttt{Rc}}
\texttt{Rc} is single-threaded reference-counting pointer. ``\texttt{Rc}'' stands for ``Reference Counted''.

\begin{minted}{rust}
    let rc = Rc::new(());
    let rc2 = rc.clone(); // Clones Rc, not what inside!
    let rc3 = Rc::clone(&rc); // The same
\end{minted}

\texttt{Rc} is dropped when all instances of \texttt{Rc} are dropped.

Primary functions:

\begin{minted}{rust}
    fn get_mut(this: &mut Rc<T>) -> Option<&mut T>;
    fn downgrade(this: &Rc<T>) -> Weak<T>;
    fn weak_count(this: &Rc<T>) -> usize;
    fn strong_count(this: &Rc<T>) -> usize;
\end{minted}
\end{frame}

% ----------------------------------------------------------------- %

\begin{frame}[fragile]
\frametitle{\texttt{Rc}}
References to the variable inside \texttt{Rc} are controlled at runtime:

\begin{minted}{rust}
    let mut rc = Rc::new(42);
    println!("{}", *rc);

    *Rc::get_mut(&mut rc).unwrap() -= 41;
    println!("{}", *rc);

    let mut rc1 = rc.clone();
    println!("{}", *rc1);
    // thread 'main' panicked at 'called `Option::unwrap()`
    // on a `None` value'
    // *Rc::get_mut(&mut rc1).unwrap() -= 1;
\end{minted}

\texttt{get\_mut} guarantees that it will return mutable reference only if there's only one pointer. If there are more, you won't have a chance to modify \texttt{Rc}.
\end{frame}

% ----------------------------------------------------------------- %

\begin{frame}[fragile]
\frametitle{\texttt{Weak}}
\texttt{Rc} is a \textbf{strong} pointer, while \texttt{Weak} is a \textbf{weak} pointer. Both of them have \textit{ownership over allocation}, but only \texttt{Rc} have \textit{ownership over the value inside}:

You can upgrade \texttt{Weak} to \texttt{Rc}:

\begin{minted}{rust}
    fn upgrade(&self) -> Option<Rc<T>>;
\end{minted}
\end{frame}

% ----------------------------------------------------------------- %

\begin{frame}[fragile,c]
\frametitle{\texttt{Weak}}
\begin{minted}{rust}
    let rc1 = Rc::new(String::from("string"));
    let rc2 = rc1.clone();
    let weak1 = Rc::downgrade(&rc1);
    let weak2 = Rc::downgrade(&rc1);
    drop(rc1); // The string is not deallocated
    assert!(weak1.upgrade().is_some());
    drop(weak1); // Nothing happens
    drop(rc2); // The string is deallocated
    assert_eq!(weak2.strong_count(), 0);
    // If no strong pointers remain, this will return zero.
    assert_eq!(weak2.weak_count(), 0);
    assert!(weak2.upgrade().is_none());
    drop(weak2); // The Rc is deallocated
\end{minted}
\end{frame}

% ----------------------------------------------------------------- %

\begin{frame}
\frametitle{Questions?}
\begin{center}
\includegraphics[width=\textwidth,height=7cm,keepaspectratio]{images/crab.jpg}
\end{center}
\end{frame}

% ----------------------------------------------------------------- %

\end{document}
