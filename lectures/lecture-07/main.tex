\documentclass[aspectratio=1610,t]{beamer}

\usepackage[english]{babel}
\usepackage{hyperref}
\usepackage{minted}
\usepackage{alltt}
\usepackage{amsmath}
\usepackage{graphicx}
\usepackage{xcolor}
\usepackage[utf8]{inputenc}
\usepackage[T1]{fontenc}

\usetheme{metropolis}
\usemintedstyle{xcode}
\definecolor{codebg}{RGB}{247, 247, 246}
\setbeamercolor{background canvas}{bg=white}
\hypersetup{colorlinks,linkcolor=,urlcolor=orange}

\title{Lecture 7: Closures}
\date{March 21, 2023}
\author{Barinov Denis}
\institute{barinov.diu@gmail.com}

\begin{document}


% ----------------------------------------------------------------- %

\begin{frame}
\maketitle
\end{frame}

% ----------------------------------------------------------------- %

\begin{frame}[c]
\centering\Huge\textbf{Closures}
\end{frame}

% ----------------------------------------------------------------- %

\begin{frame}[fragile]
\frametitle{Closures}
You've already seen closures in homeworks and lectures:

\begin{minted}{rust}
    let x = 4;
    let equal_to_x = |z| z == x;
    let y = 4;
    assert!(equal_to_x(y));
\end{minted}

\textbf{Question}: What's the difference between the closures and the functions?

\visible<2->{
    A closure is an \textit{anonymous function} that can directly \textit{use variables from the scope} in which it is defined.
}
\end{frame}

% ----------------------------------------------------------------- %

\begin{frame}[fragile]
\frametitle{Closures}
Unlike functions, closures infer input and output types since it's more convenient most of the time.

\begin{minted}[fontsize=\small]{rust}
    let option = Some(2);

    let x = 3;
    // explicit types:
    let new: Option<i32> = option.map(|val: i32| -> i32 {
        val + x
    });
    println!("{:?}", new);  // Some(5)

    let y = 10;
    // inferred:
    let new2 = option.map(|val| val * y);
    println!("{:?}", new2);  // Some(20)
\end{minted}
\end{frame}

% ----------------------------------------------------------------- %

\begin{frame}[fragile]
\frametitle{Closures and traits}
Let's try to duplicate \texttt{Option::map} functionality with handcrafted function.

\begin{minted}[fontsize=\small]{rust}
    fn map<X, Y>(option: Option<X>, transform: ...) -> Option<Y> {
        match option {
            Some(x) => Some(transform(x)),
            None => None,
        }
    }
\end{minted}

We need to fill in the \texttt{...} with something that transforms an \texttt{X} into a \texttt{Y}. What it will be?
\end{frame}

% ----------------------------------------------------------------- %

\begin{frame}[fragile]
\frametitle{Closures and traits}
We want \texttt{transform} to be the callable object. In Rust, when we want to abstract over some property, we use traits!

\begin{minted}[fontsize=\small]{rust}
    fn map<X, Y, T>(option: Option<X>, transform: T) -> Option<Y>
        where T: /* the trait */ { ... }
\end{minted}

Let's design it.

\begin{itemize}
    \item<2-> Idea: compiler generated structure that implements some trait.
    \item<3-> Our trait will have only one function.
    \item<4-> We'll use tuple as input type since we don't have variadics in Rust (and we don't actually need them, at least in this case).
\end{itemize}
\end{frame}

% ----------------------------------------------------------------- %

\begin{frame}[fragile]
\frametitle{Closures and traits}
\begin{minted}[fontsize=\small]{rust}
    trait Transform<Input> {
        type Output;
        fn transform(/* self */, input: Input) -> Self::Output;
    }
\end{minted}

\textbf{Question}: Do we need \texttt{self}, \texttt{\&mut self} or \texttt{\&self} here?

\visible<2->{
    Since the transformation should be able to incorporate arbitrary information beyond what is contained in \texttt{Input}. Without any \texttt{self} argument, the method would look like \texttt{fn transform(input: Input) -> Self::Output} and the operation could only depend on \texttt{Input} and global variables.
}
\end{frame}

% ----------------------------------------------------------------- %

\begin{frame}[fragile]
\frametitle{Closures and traits}
\textbf{Question}: What do we need exactly: \texttt{self}, \texttt{\&mut self} or \texttt{\&self}?

\visible<2->{
\begin{table}[]
\begin{tabular}{|l|l|}
\hline
                    & \textbf{User}                                \\ \hline
\texttt{self}       & Can only call method once                    \\ \hline
\texttt{\&mut self} & Can call many times, only with unique access \\ \hline
\texttt{\&self}     & Can call many times, with no restrictions    \\ \hline
\end{tabular}
\end{table}
}

\visible<3->{
    We usually want to choose the highest row of the table that still allows the consumers to do what they need to do.
}
\end{frame}

% ----------------------------------------------------------------- %

\begin{frame}[fragile]
\frametitle{Closures and traits}
Let's start with \texttt{self}. In summary, our \texttt{map} and its trait look like:

\begin{minted}[fontsize=\small]{rust}
    trait Transform<Input> {
        type Output;
        fn transform(self, input: Input) -> Self::Output;
    }

    fn map<X, Y, T>(option: Option<X>, transform: T) -> Option<Y>
        where T: Transform<X, Output = Y>
    {
        match option {
            Some(x) => Some(transform.transform(x)),
            None => None,
        }
    }
\end{minted}
\end{frame}

% ----------------------------------------------------------------- %

\begin{frame}[fragile,c]
\frametitle{\texttt{Fn}, \texttt{FnMut}, \texttt{FnOnce}}
Rust uses \texttt{Fn}, \texttt{FnMut}, \texttt{FnOnce} traits to unify functions and closures, similar to what we've invented.

\begin{minted}[fontsize=\small]{rust}
    pub trait FnOnce<Args> {
        type Output;
        fn call_once(self, args: Args) -> Self::Output;
    }

    pub trait FnMut<Args>: FnOnce<Args> {
        fn call_mut(&mut self, args: Args) -> Self::Output;
    }

    pub trait Fn<Args>: FnMut<Args> {
        fn call(&self, args: Args) -> Self::Output;
    }
\end{minted}

\visible<2->{
    Look carefuly as \texttt{self}. Every \texttt{FnMut} closure can implement \texttt{FnOnce} exactly the same way! Same applies to \texttt{Fn} and \texttt{FnMut}.
}
\end{frame}

% ----------------------------------------------------------------- %

\begin{frame}[fragile]
\frametitle{\texttt{Fn}, \texttt{FnMut}, \texttt{FnOnce}}
The real \texttt{map} looks like this:

\begin{minted}[fontsize=\small]{rust}
    impl<T> Option<T> {
        pub fn map<U, F>(self, f: F) -> Option<U>
        where
            F: FnOnce(T) -> U,
        {
            match self {
                Some(x) => Some(f(x)),
                None => None
            }
        }
    }
\end{minted}

\texttt{FnOnce(T) -> U} is another name for our \texttt{Transform<X, Output = Y>} bound, and \texttt{f(x)} for \texttt{transform.transform(x)}.
\end{frame}

% ----------------------------------------------------------------- %

\begin{frame}[fragile,c]
\frametitle{Returning and accepting closures}
Since the closure is a compiler-generated type, it's \textbf{non-denotable}, i.e you cannot write its exact type.

\begin{minted}{rust}
    fn return_closure() -> impl Fn() {
        || println!("hello world!")
    }
\end{minted}

\texttt{Fn}, \texttt{FnMut}, \texttt{FnOnce} are traits, and we can benefit from trait objects here too!

\begin{minted}{rust}
    let c1 = || {
        println!("calculating...");
        42 * 2 - 22
    };
    let c2 = || 42;
    let vec: Vec<&dyn Fn() -> i32> = vec![&c1, &c2];
    for elem in vec { println!("{}", elem()); }
\end{minted}
\end{frame}

% ----------------------------------------------------------------- %

\begin{frame}[fragile]
\frametitle{\texttt{Fn}, \texttt{FnMut}, \texttt{FnOnce}}
Basically any funtions also implement these traits!

\begin{minted}{rust}
    fn cast(x: i32) -> i64 {
        (x + 1) as i64
    }

    fn func(f: impl FnOnce(i32) -> i64) {
        println!("f(42) = {}", f(42));
    }

    fn main() {
        func(cast)
    }
\end{minted}

So, like everything in Rust, operator \texttt{()} is defined by traits
\end{frame}

% ----------------------------------------------------------------- %

\begin{frame}[fragile]
\frametitle{\texttt{fn}}
There's also \textit{function pointers} in Rust. It's not a trait, it's an actual \textit{type} that refers to the code, not data. Unlike closures, they cannot capture the environment.

\begin{minted}[fontsize=\small]{rust}
    fn add_one(x: usize) -> usize { x + 1 }

    let ptr: fn(usize) -> usize = add_one;
    assert_eq!(ptr(5), 6);

    let clos: fn(usize) -> usize = |x| x + 5;
    assert_eq!(clos(5), 10);

    // error: mismatched types
    // let y = 2;
    // let clos: fn(usize) -> usize = |x| y + x + 5;
    // assert_eq!(clos(5), 10);
\end{minted}
\end{frame}

% ----------------------------------------------------------------- %

\begin{frame}[fragile]
\frametitle{Closures: capturing}
Let's find out how Rust closures decide how to capture the variables.

\begin{minted}[fontsize=\small]{rust}
    struct T { ... }

    fn by_value(_: T) {}
    fn by_mut(_: &mut T) {}
    fn by_ref(_: &T) {}
\end{minted}
\end{frame}

% ----------------------------------------------------------------- %

\begin{frame}[fragile,c]
\frametitle{Closures: capturing}
\begin{minted}[fontsize=\small]{rust}
    let x: T = ...;
    let mut y: T = ...;
    let mut z: T = ...;

    let closure = || {
        by_ref(&x);
        by_ref(&y);
        by_ref(&z);

        // Forces `y` and `z` to be at least
        // captured by `&mut` reference
        by_mut(&mut y);
        by_mut(&mut z);

        // Forces `z` to be captured by value
        by_value(z);
    };
\end{minted}
\end{frame}

% ----------------------------------------------------------------- %

\begin{frame}[fragile]
\frametitle{Closures: capturing}
This is how closure environment will look like:

\begin{minted}[fontsize=\small]{rust}
    struct Environment<'x, 'y> {
        x: &'x T,
        y: &'y mut T,
        z: T
    }

    /* impl of FnOnce for Environment */

    let closure = Environment {
        x: &x,
        y: &mut y,
        z: z,
    };
\end{minted}
\end{frame}

% ----------------------------------------------------------------- %

\begin{frame}[fragile]
\frametitle{Closures: capturing}
Since this closure implements \texttt{FnOnce}, it cannot be called twice:

\begin{minted}{rust}
    // Ok
    closure();
    // error: moved due to previous call
    // closure();
\end{minted}
\end{frame}

% ----------------------------------------------------------------- %

\begin{frame}[fragile,c]
\frametitle{Closures: capturing}
What if you need to move out a closure from the scope? In this case, you need to move all the variables even if it's enough to have a shared reference.

\begin{minted}{rust}
    // Returns a function that adds a fixed number
    // to the argument. Reminds of Higher Order Functions
    // from functional programming!
    fn make_adder(x: i32) -> impl Fn(i32) -> i32 {
        |y| x + y
    }
    fn main() {
        let f = make_adder(3);
        println!("{}", f(1));  
        println!("{}", f(10));  
    }
\end{minted}
\end{frame}

% ----------------------------------------------------------------- %

\begin{frame}[fragile]
\frametitle{Closures: capturing}
\begin{minted}{rust}
error[E0597]: `x` does not live long enough
 --> src/main.rs:2:9
  |
2 |     |y| x + y
  |     --- ^ borrowed value does not live long enough
  |     |
  |     value captured here
3 | }
  |  -
  |  |
  |  `x` dropped here while still borrowed
  |  borrow later used here
\end{minted}
\end{frame}

% ----------------------------------------------------------------- %

\begin{frame}[fragile]
\frametitle{Closures: capturing}
Let's use \texttt{move} keyword to tell Rust we need to capture by value:

\begin{minted}{rust}
    fn make_adder(x: i32) -> impl Fn(i32) -> i32 {
        // Compiles just fine!
        move |y| x + y
    }

    fn main() {
        let f = make_adder(3);
        println!("{}", f(1));  // 4
        println!("{}", f(10));  // 13
    }
\end{minted}
\end{frame}

% ----------------------------------------------------------------- %

\begin{frame}[fragile]
\frametitle{Closures: capturing}
Going back to previous example, the closure with \texttt{move} keyword will capture all variables by value:

\begin{minted}[fontsize=\small]{rust}
    let closure = move || {
        by_ref(&x);
        by_ref(&y);
        by_ref(&z);

        // Forces `y` and `z` to be at least
        // captured by `&mut` reference
        by_mut(&mut y);
        by_mut(&mut z);

        // Forces `z` to be captured by value
        by_value(z);
    };
\end{minted}
\end{frame}

% ----------------------------------------------------------------- %

\begin{frame}[fragile]
\frametitle{Closures: capturing}
\begin{minted}{rust}
    struct Environment {
        x: T,
        y: T,
        z: T,
    }
\end{minted}

In Rust, there are no fine-grained capture lists like in C++11. \textit{But do we need it}? In practice, we don't (at least the lecturer doesn't know good examples).
\end{frame}

% ----------------------------------------------------------------- %

\begin{frame}[fragile,c]
\frametitle{Closure type}
Every closure have \textbf{distinct type}. This implies that in this example \texttt{id0}, \texttt{id1}, \texttt{id2} and \texttt{id3} have \textbf{different types}.

\begin{minted}[fontsize=\small]{rust}
    fn id0(x: u64) -> u64 { x }
    fn id1(x: u64) -> u64 { x }
    fn main() {
        let id2 = || 1;
        let id3 = || 1;
    }
\end{minted}

And this code won't compile:

\begin{minted}[fontsize=\small]{rust}
    fn make_closure(n: u64) -> impl Fn() -> u64 {
        move || n
    }

    vec![make_closure(1), make_closure(2)];
    vec![(|| 1), (|| 1)];  // Error: mismatched types
\end{minted}
\end{frame}

% ----------------------------------------------------------------- %

\begin{frame}[fragile]
\frametitle{Closures and optimizations}
\begin{itemize}
    \item<1-> We create a structure for the closure, do some moves... It must be expensive!
    \item<2-> Actually, the compiler knows a lot about our code and optimizes it with ease. Most closure calls are inlined and in binary is the same as code without closure.
    \item<3-> Zero cost abstraction!
\end{itemize}
\end{frame}

% ----------------------------------------------------------------- %

\begin{frame}
\frametitle{Questions?}
\begin{center}
\includegraphics[width=\textwidth,height=7cm,keepaspectratio]{images/crab.jpg}
\end{center}
\end{frame}

% ----------------------------------------------------------------- %

\end{document}
